\documentclass[12pt]{article}


%%%%%%%%%%%%%%%%%%%%%  MY STUFF %%%%%%%%%%%%%%%%%%%%%%%%%%%%%%%%%%

%\usepackage{fancyhdr}
%\pagestyle{fancy}
%\fancyfoot[C]{ {\bf Page \thepage\ of ??? -- ADV. ECON. METHODS (EMET3011/8014)}}
%\fancyhead{}
%\renewcommand{\headrulewidth}{0pt}
%\addtolength{\footskip}{80pt}

\usepackage{amsmath, amssymb, amsthm}
\usepackage{mathpazo} 
\usepackage[T1]{fontenc}

\usepackage[utf8]{inputenc}

% caligraphic
\usepackage{mathrsfs}
\usepackage{bbm}
\usepackage{subfigure}

\usepackage{enumerate}

\renewcommand{\baselinestretch}{1.1}


\setcounter{tocdepth}{1}

% skip a line between paragraphs, no indentation
\setlength{\parskip}{1.5ex plus0.5ex minus0.5ex}
\setlength{\parindent}{0pt}

\newcommand{\navy}[1]{\textcolor{Blue}{\bf #1}}

\newcommand{\argmax}{\operatornamewithlimits{argmax}}
\newcommand{\argmin}{\operatornamewithlimits{argmin}}

\DeclareMathOperator{\cl}{cl}
\DeclareMathOperator{\se}{se}
%\DeclareMathOperator{\argmax}{argmax}
\DeclareMathOperator{\interior}{int}
\DeclareMathOperator{\Prob}{Prob}
\DeclareMathOperator{\determinant}{det}
\DeclareMathOperator{\Span}{span}
\DeclareMathOperator{\rank}{rank}
\DeclareMathOperator{\range}{rng}
\DeclareMathOperator{\trace}{trace}
\DeclareMathOperator{\cov}{cov}
\DeclareMathOperator{\corr}{corr}
\DeclareMathOperator{\var}{var}
\DeclareMathOperator{\mse}{mse}

\newcommand{\ess}{ \textrm{{\sc ess}} }
\newcommand{\tss}{ \textrm{{\sc tss}} }
\newcommand{\ssr}{ \textrm{{\sc ssr}} }
\newcommand{\rssr}{ \textrm{{\sc rssr}} }
\newcommand{\ussr}{ \textrm{{\sc ussr}} }


% mics short cuts and symbols
\newcommand{\st}{\ensuremath{\ \mathrm{s.t.}\ }}
\newcommand{\setntn}[2]{ \{ #1 : #2 \} }
\newcommand{\fore}{\therefore \quad}
\newcommand{\tod}{\stackrel { d } {\to} }
\newcommand{\eqdist}{\stackrel {\textrm{ \scriptsize{d} }} {=} }
\newcommand{\iidsim}{\stackrel {\textrm{ {\sc iid }}} {\sim} }
\newcommand{\1}{\mathbbm 1}
\newcommand{\dee}{\,{\rm d}}
\newcommand{\given}{\, | \,}
\newcommand{\la}{\langle}
\newcommand{\ra}{\rangle}

\newcommand{\boldx}{ {\bf x} }
\newcommand{\boldw}{ {\bf w} }
\newcommand{\boldu}{ {\bf u} }
\newcommand{\boldy}{ {\bf y} }
\newcommand{\boldb}{ {\bf b} }
\newcommand{\bolda}{ {\bf a} }
\newcommand{\boldi}{ {\bf i} }
\newcommand{\bolde}{ {\bf e} }
\newcommand{\bolds}{ {\bf s} }
\newcommand{\boldz}{ {\bf z} }
\newcommand{\boldv}{ {\bf v} }

\newcommand{\boldzero}{ {\bf 0} }
\newcommand{\boldone}{ {\bf 1} }

\newcommand{\boldalpha}{ {\boldsymbol \alpha} }
\newcommand{\boldbeta}{ {\boldsymbol \beta} }
\newcommand{\boldgamma}{ {\boldsymbol \gamma} }
\newcommand{\boldtheta}{ {\boldsymbol \theta} }
\newcommand{\boldepsilon}{ {\boldsymbol \epsilon} }
\newcommand{\boldSigma}{ {\boldsymbol \Sigma} }

\newcommand{\hboldy}{ \hat {\bf y} }
\newcommand{\hboldu}{ \hat {\bf u} }
\newcommand{\hboldbeta}{ \hat {\boldsymbol \beta} }

\newcommand{\boldA}{\mathbf A}
\newcommand{\boldZ}{\mathbf Z}
\newcommand{\boldB}{\mathbf B}
\newcommand{\boldC}{\mathbf C}
\newcommand{\boldD}{\mathbf D}
\newcommand{\boldM}{\mathbf M}
\newcommand{\boldP}{\mathbf P}
\newcommand{\boldI}{\mathbf I}
\newcommand{\boldX}{\mathbf X}
\newcommand{\boldY}{\mathbf Y}


\newcommand{\RR}{\mathbbm R}
\newcommand{\NN}{\mathbbm N}
\newcommand{\PP}{\mathbbm P}
\newcommand{\EE}{\mathbbm E \,}
\newcommand{\XX}{\mathbbm X}
\newcommand{\ZZ}{\mathbbm Z}
\newcommand{\QQ}{\mathbbm Q}

\newcommand{\fF}{\mathcal F}
\newcommand{\nN}{\mathcal N}


\newcommand{\Xsf}{\mathsf X}
\newcommand{\Zsf}{\mathsf Z}


\theoremstyle{definition}
\newtheorem{question}{Question}



%%%%%%%%%%%%%%%%%% end my preamble %%%%%%%%%%%%%%%%%%%%%%%%%%%%%%%%%%%


\begin{document}


\title{Practice Questions}

\date{}

%\maketitle

\begin{center}

    {\bf {\Large Introduction to Computational Macroeconomics}}

    \bigskip
    {\bf {\Large Tokyo 2022}}

    \bigskip
    {\bf {\Large Assignment}}

    \today
\end{center}

\begin{itemize}
    \item Due date: 12th July
    \item Weight: 40\% of total marks
\end{itemize}

This assignment should be submitted by email as a Jupyter notebook no later
than midnight on the due date.  (Points will be deducted for late
assignments.)  Please send to \texttt{john.stachurski@anu.edu.au}.  

Note that proofs and other discussion should also be included in the notebook.
If you are not sure how to do this please ask.  No other files should be
submitted.  I recommend that you stop and restart Jupyter before you submit,
and make sure that the notebook runs from start to finish without error.

\vspace{1em}

Consider a worker who is currently employed and considering when to retire.
While employed, the worker is paid a fixed wage $w$ in each period.  When she
retires, the worker is paid a single lump sum pension $s$.  Both $w$ and $s$
are nonnegative and constant.  (In particular, wages are not indexed to
inflation.)

The worker discounts future earnings using the real interest rate $r_t = i -
\pi_t$, where $i$ is the nominal risk-free interest rate.  In other words, a
payment $y_{t+1}$ is discounted to time $t$ present value via $\beta_t
y_{t+1}$, where $\beta_t = 1/(1 + r_t)$.  The nominal risk-free rate is
constant.

Assume that the inflation rate obeys $\pi_t = \pi(Z_t)$ for some fixed
function $\pi \in \RR^\Zsf$, where $\Zsf$ is a finite subset of $\RR$ and
$(Z)_{t \geq 0}$ is $Q$-Markov for some stochastic matrix $Q$ on $\Zsf$.
In what follows we set $n := |\Zsf|$.


\begin{question}
    Write down the Bellman equation for the problem, in terms of the lifetime
    value for an employed worker.  Define the Bellman operator $T$
    corresponding to the Bellman equation as a self-map on $\RR^\Zsf$. 
\end{question}

\begin{question}
    Prove that $T$ is a contraction map on $\RR^\Zsf$ with respect to the
    supremum norm whenever $\bar r := \min_{z \in \Zsf} (i - \pi(z))$
    satisfies $\bar r > 0$.
\end{question}

Although the condition $\bar r > 0$ is sufficient for global stability of $T$,
it is not necessary.  As we will learn later, it sufficies that the matrix
$n \times n$ matrix
%
\begin{equation*}
    L(z, z') := \frac{1}{1 + i - \pi(z)} Q(z, z')
\end{equation*}
%
satisfies $r(L) < 1$.  

\begin{question}
    Let $\pi(z) = p + z$, where $p$ is a parameter indicating the long-run
    average inflation rate.  Show that $r(L) < 1$ holds when $(Z_t)$ is
    the Tauchen discretization of the AR(1) process $X_{t+1} = \rho X_t + \nu
    W_{t+1}$, where $(W_t)_{t \geq 0}$ is {\sc iid} standard normal and
    the parameters are given by
    %
    \begin{equation}\label{eq:dp}
        s = 100, \;
        w = 5,   \;
        i = 0.1,\;
        p = 0.06,\;
        \rho = 0.9 \; \text{and} \;
        \nu = 0.01
    \end{equation}
    %
    The AR(1) process should be discretized to a grid of $n = 100$ points
    (which is the size of the state space $\Zsf$).  To discretize, 
    you can use the commands \texttt{import quantecon as qe} and
    %
    \begin{center}
        \texttt{mc = qe.tauchen($\rho$, $\nu$, m=10, n=n)}
    \end{center}
\end{question}

\begin{question}
    Solve for the value function $v^*$ using value function iteration using
    the default parameters in \eqref{eq:dp}.  Plot $v^*$ as
    a function of $z$. Plot the stopping value $s$ on the same figure.    
\end{question}

\begin{question}
    Plot the $v^*$-greedy policy as a function of $z$.
    Provide some interpretation of the figure.  
\end{question}

\begin{question}
    Prove that the value function $v^*$ is increasing on $\Zsf$ under the
    parameters in \eqref{eq:dp}.  Provide some economic intuition if you can.
\end{question}

\begin{question}
    Using the default parameters and simulation, compute the average number of
    years until retirement at the default parameters.  In answering this
    question, you should regard one period in the model as one month (i.e.,
    wages are paid monthly and interest and inflation rates are monthly).
    If you can, accelerate your simulation using Numba.
\end{question}

\begin{question}
    Now set $w=4.9$ and repeat the last exercise.  What change do you observe?
    How can you explain this change?
\end{question}


\end{document}













